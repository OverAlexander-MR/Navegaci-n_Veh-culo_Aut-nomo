\documentclass{beamer}
\usetheme{Madrid}
\usepackage[utf8]{inputenc}
\usepackage[spanish]{babel}
\usepackage{graphicx}

\title{Sistema de Navegación Adaptativa para un Vehículo Autónomo}
\author{
    Over Alexander Mejía Rosado \\
    Ronald Mateo Ceballos Lozano \\
    Rhonald José Torres Díaz
}
\institute{\textit{Inteligencia Artificial} \\
Universidad Nacional de Colombia - De La Paz}
\date{}

\begin{document}

\begin{frame}
    \titlepage
\end{frame}

\begin{frame}{Introducción}
    \begin{itemize}
        \item Alto índice de accidentes en el tránsito terrestre debido a errores humanos.
        \item Crecimiento del número de vehículos a nivel mundial.
        \item Necesidad de sistemas de automatización y tráfico inteligente.
    \end{itemize}
\end{frame}

\begin{frame}{Objetivo del Proyecto}
    \begin{itemize}
        \item Diseñar e implementar un vehículo autónomo capaz de desplazarse en un mapa 2D.
        \item Utilizar NEAT (NeuroEvolution of Augmenting Topologies) para desarrollar el sistema de navegación.
        \item Mejorar la capacidad del vehículo de adaptarse a condiciones variables.
    \end{itemize}
\end{frame}

\begin{frame}{Metodología}
    \begin{itemize}
        \item Implementación de sensores virtuales para la percepción del entorno.
        \item Uso de métricas de distancia (Euclidiana, Manhattan, Chebyshev) como sistema de recompensas.
        \item Creación de un sistema de refuerzo forzado para evitar que el vehículo se estanque.
    \end{itemize}
\end{frame}

\begin{frame}{NEAT (NeuroEvolution of Augmenting Topologies)}
    \begin{itemize}
        \item Algoritmo evolutivo para el desarrollo de redes neuronales.
        \item Evoluciona tanto los pesos como la topología de las redes.
        \item Permite la adaptación y optimización continua del vehículo autónomo.
    \end{itemize}
\end{frame}

\begin{frame}{Resultados}
    \begin{itemize}
        \item El vehículo autónomo es capaz de desplazarse en el mapa de forma autónoma.
        \item Incremento del fitness en cada generación según la métrica de distancia utilizada.
        \item Mejora en la adaptabilidad y eficiencia del sistema de navegación.
    \end{itemize}
\end{frame}

\begin{frame}{Conclusiones}
    \begin{itemize}
        \item El uso de NEAT es efectivo para desarrollar sistemas de navegación adaptativos.
        \item Las métricas de distancia y el refuerzo forzado contribuyen al aprendizaje del vehículo.
        \item Potencial para reducir accidentes y mejorar la eficiencia en sistemas de transporte autónomo.
    \end{itemize}
\end{frame}

\begin{frame}{Preguntas}
    \centering
    {\Huge ¿Preguntas?}
\end{frame}

\end{document}